%!TEX root = ../main.tex

\documentclass[../main.tex]{subfiles}

\begin{document}

\chapter{Background}
\label{chapter:background}


\begin{figure}
	\centering
	\subfile{img/chapter_2/workspace_and_system}
	\caption{The workspace polygon and a point model of the robot.}
	\label{fig:workspace_and_system}
\end{figure}

\begin{figure}
	\centering
	\subfile{img/chapter_2/configuration_space}
	\caption{An example of a configuration space for a point robot.}
	\label{fig:configuration_space}
\end{figure}

\begin{figure}
	\centering
	\subfile{img/chapter_2/coverable_space}
	\caption{An example of a coverable space.}
	\label{fig:coverable_space}
\end{figure}

\begin{definition}[Minkowski Sum]
Given two sets $A$ and $B$, a Minkowski sum is a set $C=\{a+b\ |\ a\in A, b\in B\}$.
\end{definition}

\begin{definition}[Workspace]
The workspace $W\in\mathbb{R}^n$ is a region where the robot tasked with coverage will operate. The workspace $W$ is connected and may be convex or non-convex. In this paper, we will sometime constrain $W$ to be polygonal. An example of such workspace is demonstrated in Figure~\ref{fig:workspace_and_system}.
\end{definition}

\begin{definition}[Obstacles]
A workspace may contain $k$ obstacles, $O_1,\ldots,O_k$. These are subregions contained entirely in the workspace that are not reachable by the robot. 
\end{definition}

\begin{definition}[Free Workspace]
A free workspace is defined as a subset of $W$ that is free of obstacles:
	\begin{equation}
	\begin{aligned}
		W_{\text{free}}=W\setminus(O_1\cup\dots O_n).
	\end{aligned}
	\end{equation}
\end{definition}

\begin{definition}[Robot Configuration Space and Map]
\label{definition:c_space_and_map}
Given a robot with dynamics, the configuration space of the robot is $Q$. A configuration map $\mathcal{B}:Q\to O\subseteq W$ is a function mapping the configuration $q\in Q$ of the robot to a set of points in $W$ located under the robot's \emph{footprint}. For example, in Figure~\ref{fig:configuration_space}, the footprint is the orange disk centered at $(x,y)$ and the set $O$ is all points under the orange disk.
\end{definition}

\begin{definition}[Robot Location Map]
\label{definition:location_map}
Given a robot with dynamics, the configuration space of the robot is $Q$. A location map $\mathcal{L}:Q\to W$ is a function mapping the configuration $q\in Q$ of the robot to a \emph{center} point of the robot. For example, in Figure~\ref{fig:configuration_space}, the location of the center of the robot is $(x,y)$.
\end{definition}

\begin{definition}[Free Configuration Space]
\label{definition:free_c_space}
The free configuration space is defined as:
	\begin{equation}
	\begin{aligned}
		Q_{\text{free}}=\{q\in Q\ |\ \mathcal{B}(q)\ \text{is inside}\ W_{\text{free}}\}.
	\end{aligned}
	\end{equation}
\end{definition}

\begin{definition}[Coverage Map]
A coverage map is a function $\mathcal{M}:Q\to F\subseteq W$ mapping the configuration $q\in Q$ of the robot to a set of points in $W$ located under the robot's \emph{coverage footprint}. For example, in Figure~\ref{fig:workspace_and_system}, the coverage footprint is the green disk centered at $(x,y)$ and the set $F$ is all points under the green disk.
\end{definition}

\begin{definition}[Coverable Space]
The coverable space $\mathcal{C}$ is a set of points in the free workspace reachable by the coverage footprint as shown in Figure~\ref{fig:coverable_space}. Formally,
	\begin{equation}
	\begin{aligned}
		\mathcal{C}=\bigcup_{q\in Q_{\text{free}}}\mathcal{M}(q).
	\end{aligned}
	\end{equation}
\end{definition}

\begin{definition}[Set of Feasible Paths]
A set of feasible paths is defined as
	\begin{equation}
	\begin{aligned}
		P=\{p_i\ |\ p_i\subseteq Q_{\text{free}}\}.
	\end{aligned}
	\end{equation}
\end{definition}

\begin{definition}[Path Metric]
A path metric is a function $\mathcal{E}:P\to\mathbb{R}$ that assign a cost to a feasible path. This cost is usually determined by the application.
\end{definition}

\begin{definition}[Straight Line Segment]
A straight line segment, $s$, is a path consisting of points in a straight line. A robot travelling in a straight segment will not change its heading. i.e.
	\begin{equation}
		s=\{\theta x+(1-\theta)y\in W\ |\ \theta\in[0,1],\ x,y\in W\}.
	\end{equation}
A set of straight line segments in path is $S$.
\end{definition}

\begin{definition}[Transition Segment]
A transition segment, $t$, is a path connecting two straight line segments and a robot traversing a transition segment must have a rate of change of the heading greater than zero. A set of transition segments in a path is $T$.
\end{definition}

Riemann sum

\subsection{Geometric Preliminaries}
\begin{definition}[Simple Polygon]
A \emph{simple polygon} is a non-intersecting chain of straight line segments forming a closed loop and specified as a set of points, i.e.
	\begin{equation}
	\begin{aligned}
		Z=\{v_i\in\mathbb{R}^2|i=1,\ldots,n\}.
	\end{aligned}
	\end{equation}
Note that since the chain is a circuit, any $v_i\in Z$ has two adjacent line segments. 
\end{definition}

\begin{definition}[Simple Polygon Boundary]
A \emph{boundary of a simple polygon} is a set of points, $\partial Z$, along a line connecting any two adjacent vertices of a chain. 
\end{definition}

\begin{definition}[Clockwise Simple Polygon]
A \emph{clockwise simple polygon} is a simple polygon where vertices are specified in a clockwise order. We associate these types of simple polygons with holes in the workspace. 
\end{definition}

\begin{definition}[Counter-clockwise Simple Polygon]
A \emph{counter-clockwise simple polygon} is a simple polygon where vertices are specified in a counter-clockwise order. This type of simple polygons are associated with the external boundary of the workspace.
\end{definition}

\begin{definition}[Polygon]
A \emph{polygon} is a set of clockwise and counter-clockwise simple polygons. For clarity, we will refer to simple polygons as chains and reserve the term polygon for a set of chains, i.e., $P=\{Z_0,\ldots,Z_m\}$. $Z_0$ is a counter-clockwise chain (i.e., the boundary) and all subsequent $Z_i$ are clockwise chains (i.e., holes). 
\end{definition}

\begin{definition}[Polygon Boundary]
A \emph{boundary of a polygon} is the set of points defined as
	\begin{equation}
	\begin{aligned}
 		\partial P=\bigcup^M_{i=1}\partial Z_i
	\end{aligned}
	\end{equation}
where $Z_i\in P$.
\end{definition}

\begin{definition}[Reflex Vertex]
A \emph{reflex vertex} is a vertex in one of the chains of $P$ that has an internal angle of more than $\pi$.
\end{definition}

\begin{definition}[Polygon Convexity]
A polygon is \emph{convex} if and only if there are no reflex vertices. Conversely, a polygon is non-convex if and only if it contains at least one reflex vertex. Note that the presence of a hole guarantees at least one reflex vertex.
\end{definition}

\subsection{Generalized Traveling Salesman Problem Preliminaries}
Suppose a complete graph $G=(V,E,w)$ is given where $V$ is a set of vertices, $E$ is a set of edges, and $w$ is a set of edge weights. Suppose $V$ is partitioned into pairwise disjoint sets $\{V_1,\ldots,V_p\}$ where $V_i\subset V$. The Generalized Traveling Salesman Problem (GTSP) is the problem of computing a tour that visits exactly one vertex from each $V_i$ such that the length of the tour is minimized. The TSP is a special case of the GTSP where $|V_i|=1$ for each $i$ and is NP-hard.

\begin{definition}[Altitude]

\end{definition}

\begin{definition}[Cone of bisection]
\label{def:cone_of_bisection}
Suppose a non-convex polygon $P$ containing a reflex vertex $v$ is given. Let $d_1=\{v_k, v\}$ and $d_2=\{v, v_l\}$ be the two adjacent edges of $v$. The \emph{cone of bisection} at $v$ is defined by two line segments, $e_1=\{v, w_1\}$ and $e_2=\{v, w_2\}$ that are parallel to $d_1$ and $d_2$ respectively with $w_1, w_2\in{\partial P\setminus\{v\}}$. See Figure%~\ref{fig:cut}.
\end{definition}

\begin{definition}[Cut space]
\label{def:cut_space}
Consider a non-convex polygon $P$ with reflex vertex $v$, along with the cone of bisection. The cut space is a set $S\subset\partial P$ of all points on $\partial P$ between $w_1$ and $w_2$ visible from $v$.
\end{definition}

\begin{definition}[Decomposing Cut]
\label{def:decomposing_cut}
Given a non-convex polygon $P$ with a reflex vertex $v$, a decomposing cut is a straight line segment $e=\{v,w\}$ within the cone of bisection of $v$ where $v,w$ belong to the same chain. See Figure~%\ref{fig:cut}.
\end{definition}

Medial axis
Contours
%\onlyinsubfile{
	%\bibliographystyle{IEEEtran}
	%\bibliography{../main}
%}
\end{document}