%!TEX root = ../main.tex

\documentclass[../main.tex]{subfiles}
\begin{document}


\chapter{Literature Review}
\label{chapter:literature_review}

The problem of coverage path planning sees many applications across numerous fields. The list includes applications like material polishing~\cite{rososhansky2010coverage}, autonomous farming~\cite{ollis1997vision}, demining~\cite{acar2003path}, floor cleaning~\cite{yasutomi1988cleaning}, and many others. The roots of this problem could be traced to CNC trajectory planning where operators rely on their experience and intuition to plan an optimal trajectory for a tool to cut out different shapes out of a slab of metal. Perhaps one of the earliest works in literature is the work by Cao~\cite{cao1988region} where the authors outline the requirements for a coverage path. These conditions state that all points in the workspace must be covered, there should be no footprint overlap, all obstacles are to be avoided, and optimality is desired. These conditions are strict in that it is not always possible to fully satisfy them. However, this work has set the stage for the future research.

One of the key contributions to the field of coverage is the work by Arkin~\cite{arkin2000approximation}. The authors have studied two variations of the coverage problem: lawn mowing and milling. The difference is that the milling problem does not allow a robot to cross the boundary of its workspace. The authors were able to show that these problems are in fact NP-hard, which means the computational time to solve it increases dramatically with the size of the input. Some of the other problems related to coverage are the Travelling Salesman Problem, art gallery problem, and a watchman route problem.

There are various classifications used when talking about coverage problems. Choset~\cite{choset2000coverage} have classified majority of the approaches by the methods used to discretized the environment. He summarized these to be either exact or approximate. However, since his survey, numerous new works have been published that do not strictly fall into these two categories. The new categories would include random and pseudo-random approaches, 3D coverage, coverage under uncertainty, persistent coverage, cooperative coverage, and others. In this section, we aim to review some of the important works involving exact and approximate decompositions as well as works that deal with the cooperative coverage.

\section{Exact Cellular Decomposition}
\label{section:exact_cellular_decomposition}

Typically, approaches that rely on exact cellular decomposition compute a solution in two steps. The first step involves performing some sort of decomposition of the workspace resulting in a set of cells that together make up the original workspace. The second step involves computing a path for each region separately. One of the classic examples of this is the work by Latombe~\cite{latombe1991exact}, where a trapezoidal decomposition is performed on a polygonal planar workspace. The resulting decomposition consists of trapezoidal cells. An adjacency graph can be computed for these cells where an edge between two vertices exists if two cells are adjacent. A tour of this graph is computed where each cell is visited once. The coverage of each cell is by a \emph{zig-zag} like path.  In practice, this method was used on an agricultural field by Oksanen~\cite{Oksanen2009coverage}. Despite the simplicity of its computation, there are numerous drawbacks to this method. One of them is the excess of cells generated, which leads to a significant overhead associated with transitioning between cells. Another problem is the direction of these zig-zag paths. This direction is determined by the decomposition, which disregards the optimality of coverage of that cell. Another problem is the generality of this method with respect to the shape of the footprint. For example, a circular footprint would not be able to cover areas near the corners. As such, the path generated by this method would not be able to achieve complete coverage.

To mitigate the problem of excessive number of cells, Choset~\cite{choset2000coverage} proposed a new decomposition technique called the Boustrophedon decomposition. The Boustrophedon decomposition is very similar to the trapezoidal decomposition with the exception that some cells in the decomposition may be non-convex and the overall number of cells in the decomposition is reduced. The path planning components is identical to Latombe's approach and therefore, suffers from similar drawbacks.

Wong~\cite{wong2004complete}'s landmark based decomposition is similar to the Boustrophedon but works for general workspaces. His approach is also suitable for on-line sensor based coverage. However, the individual cells are covered via the same zig-zag type paths.

Acar~\cite{Acar2002morse} proposed a more general decomposition technique that works on non-polygonal workspaces and could generate cells of various shapes to accommodate any coverage requirements. This decomposition is based on finding critical points of a Morse function. Boustrophedon decomposition is a specific case of a Morse decomposition where the Morse function is a straight line. Moreover, the search for these critical points can be performed on-line leading to a number of on-line coverage algorithms~\cite{acar2002sensor}. The main drawback of this approach is that rectilinear workspace cannot be used. 

Huang~\cite{Huang2001optimal} studied an approach to decomposition that aims to optimize the zig-zag paths within each cell. In his work, he proposed a dynamic programming approach for choosing shapes for the cells of the decomposition in a way that minimizes the number of parallel segments of a zig-zag path within each cell. However, the transitions costs between cells are completely ignored.

Recently, in the work by Das at el.~\cite{das2014mapping}, a greedy decomposition of a polygonal environment was proposed that seems to result in fewer convex cells for typical workspace polygons. As with other approaches, a tour of the graph is computed. However, the inter-cell coverage of each individual cell is computed based on the graph tour. Specifically, the entry and exit points are chosen in a way that minimizes the transition costs between cells. As a result, the optimality of the intra-cell coverage within the cell is ignored.

There is a common underlying problem with exact decomposition methods. The problem of convex decomposition is known to be NP-hard for general polygons as outlined in the survey by Keil~\cite{keil2000polygon}. As such, most of the exact decompositions algorithms utilize heuristics. Moreover, there is a inter-dependency between inter-cell and intra-cell coverage. In other words, a good solution cannot focus on optimizing the intra-cell coverage but ignoring the inter-cell transitions or vice versa.

\section{Grid Based Approaches}
\label{section:grid_based_approaches}

Exact cellular decompositions is not the only available class of solutions. Another class includes grid-based methods, sometimes referred to as approximate decomposition methods. Typically, the workspace is represented as a grid of uniform cells. Such representation is easy to compute but the memory requirement scales exponentially with the size of the workspace.

Zelinsky~\cite{Zelinsky1993planning} proposed a grid based coverage solution based on a wave propagation algorithm. With this method, a start and a goal cells are given. A front is created by computing the distance from the goal to the cell. Coverage is ensured by traversing the wave front, a set of cells where the values are equal. On-line generalization is available by Shrivashakar et al.~\cite{shivashankar2011real}. The work by Gabriely and Rimon~\cite{Gabriely2001spanning} is based on a spanning tree coverage. The grid of the workspace acts as a graph were every cell is a node connected to its neighbors. A minimum spanning tree can be computed for slightly modified version of this graph. A coverage path is generated by tracing the edges of the minimum spanning tree, which ensures all cells are covered at least once. Arkin~\cite{arkin2000approximation} also proposed approximate solutions based on a grid decomposition of the workspace. The workspace is represented in the form of a grid where the center point is a node in a graph. Such graph is fed into a Travelling Salesman Problem solver and the shortest tour is compute.

All of the mentioned approaches share similar downsides. The representation of the workspace by a grid is approximate. Hence, it is expected to see some points near corners or the boundary not being covered. Furthermore, computational load for large environment makes these methods intractable. Furthermore, these methods suffer from generalization problem as well. Grid based approaches do not work well with circular coverage footprints. A robot visiting two adjacent grid cells would have to incur some overlap because the circle shape has to be larger than the grid cell. This leads to the lose of efficiency of the coverage path and results in longer paths.


\section{Minimizing Turns in Coverage Paths}
\label{section:minimizing_turns_in_coverage_paths}

So far the works considered here do not consider dynamics of a robot. Many of the proposed solutions in literature produce paths that may not be feasible for the types of robots that perform coverage. A common way to implicitly compute a path that minimizes the dynamic cost of a coverage path is by minimizing the number of turns in a coverage path. 

The work by Huang~\cite{Huang2001optimal} attempted to solve this by proposing a decomposition scheme via dynamics programming. With this convex decomposition, the number of cells for the overall coverage is minimized. Arkin~\cite{arkin2005optimal} have studied a TSP like solution with turning cost embedded into the cost. An approximate solution is proposed. Wagner~\cite{wagner2001approximation} proposed another approximate solution with bounds. However, these two solutions restrict the type of coverage path to be rectilinear, which is overly restrictive in practice.


\section{Multi-agent Coverage}
\label{section:multi_agent_coverage_lit_review}

Decreasing cost of robotics motivates the use of multi-agent systems to tackle some task. Utilizing a team of robots rather than a single robot has multiple benefits, one of which is the ability to complete a task faster. Recently, there have been a surge of works in literature for tackling robot collaboration in common tasks. For coverage, a common approach is to extend the concepts developed for the single agent coverage by computing a work partition. In this work partition, each robot is assigned its own region. For each such region, a single agent coverage problem is solved.

The work by Durham~\cite{durham2012discrete} outlines an interesting solution to the area partitioning problem. The paper tackles a problem of the multi-agent coverage problem under unreliable communication network. The notion of coverage here is different in that a robot's sensor's ability to cover cells is a function of the distance. The environment is modeled as a grid with obstacles where the robots can move from one cell to an adjacent cell in the grid. The objective is to compute a partition that minimizes the coverage cost function. The computation of this function shares some similarities to Voronoi partitioning. The partitioning algorithm relies on pair-wise partitioning and reaches pair-wise optimality. The problem with this approach is that the environment as modeled as an occupancy grid. 
%Also, this approach differs from ours as the robots is not requires to visit all cells; therefore, the cost function used in the optimization of the partition reflects that.

The work by Maza~\cite{maza2007multiple} proposed a solution to multi-agent coverage. Given $n$ robots, the polygonal workspace is partitioned into $n$ regions. This partition is performed via an anchored area partitioning algorithm. The area of the cells in this partition reflects each robot's individual abilities. Each cell is covered in an optimal manner. The problem is that the authors only consider convex polygons. Furthermore, although the abilities of the robots can be approximated with area, it seems that there should be better approximations. This approach was validated with extensive trials in~\cite{barrientos2011aerial}.

%considered a similar problem with aerial drones. The solution is computed in two steps. First is the task allocation and the second is the path planning for each individual drone. The subregions are represented as a grid. This work shares a lot of similarities to the previous author but with extensive test trials.

Hokeyam~\cite{hokayem2007dynamic} worked on an algorithmic approach to coverage by robots following inner level sets of the polygonal workspace. These inner level sets are computed by shrinking the boundary of the polygon inwards by a fixed distance. A team of robots follow these level sets and resolve any collisions via sporadic communications. The idea of inner level sets is appealing for convex polygons. However, for non-convex polygons, the transitions between inner level sets could be tricky. This issue was not addressed by the authors. Moreover, this method of coverage usually ends up with excessive number of turns.

In another work~\cite{atincc2013supervised}, the problem of coverage was studied as a control problem. The coverage control law is built based on a gradient decent method with the addition of a transition controller to help the robot escape local minima of the gradients. 

Another work by Valente~\cite{valente2011multi} and Barrientos~\cite{barrientos2011aerial} propose a solution to the multi-agent problem. The solution involves several steps. The workspace is partitioned into cells with areas specified by the robots' abilities. This is performed in a distributed manner via a negotiation process. The resulting cells are discretized into a grid and a flight path is computed for each cell individually. The work by Barrientos contain extensive field tests with a team of UAVs.


%The generation of an optimal coverage path for convex work spaces is tractable and can be solved efficiently using sweeping or spiralling motions. Minimum turn coverage for convex polygons has been considered in~\cite{maza2007multiple}. For general workspaces, the coverage problem is NP-hard~\cite{arkin2000approximation}. However, there are many approximate and heuristic approaches to coverage. One class of approaches is an \emph{exact} convex decomposition on the workspace in which the workspace is partitioned into convex regions. A coverage path consists of a tour of each region, with local coverage of a convex region performed with either sweeping or spiralling motion. Commonly-used methods for decomposition include Boustrophedon decomposition~\cite{Choset1998coverage}, its more general form, Morse-based decomposition~\cite{Acar2002morse}, and trapezoidal decomposition~\cite{Oksanen2009coverage}. Huang~\cite{Huang2001optimal} proposed a dynamic programming approach for generating a set of regions that minimize the number of turns in the overall coverage path. However, the runtime is exponential, and the cost associated with transitions between these regions is assumed to be negligible. Das~\cite{das2014mapping} used a greedy cut method for convex decomposition and computed a tour of the regions using a Traveling Salesman Problem (TSP) solver.

%With exact decomposition, the quality of the coverage path is highly-dependent on the convex decomposition. For polygons without holes (obstacles), there are several suboptimal and optimal decomposition algorithms available~\cite{keil2000polygon}. Convex decomposition for polygons with holes is NP-hard~\cite{lingas1982power}. A drawback of exact decomposition is that for complex workspaces, the decomposition may contain many regions, resulting in large total transition costs between regions.



%Another class of approaches is \emph{approximate} convex decomposition~\cite{Galceran2013surveycpp}, in which convex regions may overlap, and the union of regions approximates the workspace (points left uncovered are typically handled in a second pass). In the most common approach, each of the overlapping region is a coverage footprint, as shown in Figure~\ref{fig:footprints}(a). Local coverage of each region is ensured by visiting the region's center. Complete coverage becomes the problem of planning a tour that visits each center with minimum cost~\cite{arkin2000approximation}. A drawback of this method is that the overlap between regions reduces the efficiency of the overall coverage path. Also, the number of regions grows with the area of the workspace, making the problem of computing tours of the centers intractable for large spaces.

%There are many works that deal with multi-agent coverage. The concept of area coverage could be divided into classes. One has to be careful navigating the literature to pick the right class of coverage.



\end{document}