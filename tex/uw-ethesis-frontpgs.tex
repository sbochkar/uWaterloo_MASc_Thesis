% T I T L E   P A G E
% -------------------
% Last updated May 24, 2011, by Stephen Carr, IST-Client Services
% The title page is counted as page `i' but we need to suppress the
% page number.  We also don't want any headers or footers.
\pagestyle{empty}
\pagenumbering{roman}

% The contents of the title page are specified in the "titlepage"
% environment.
\begin{titlepage}
        \begin{center}
        \vspace*{1.0cm}

        \Huge
        {\bf Minimizing Turns in Single and Multi Robot Coverage Path Planning}

        \vspace*{1.0cm}

        \normalsize
        by \\

        \vspace*{1.0cm}

        \Large
        Stanislav Bochkarev \\

        \vspace*{3.0cm}

        \normalsize
        A thesis \\
        presented to the University of Waterloo \\ 
        in fulfillment of the \\
        thesis requirement for the degree of \\
        Master of Applied Science \\
        in \\
        Electrical and Computer Engineering \\

        \vspace*{2.0cm}

        Waterloo, Ontario, Canada, 2016 \\

        \vspace*{1.0cm}

        \copyright\ Stanislav Bochkarev 2016 \\
        \end{center}
\end{titlepage}

% The rest of the front pages should contain no headers and be numbered using Roman numerals starting with `ii'
\pagestyle{plain}
\setcounter{page}{2}

\cleardoublepage % Ends the current page and causes all figures and tables that have so far appeared in the input to be printed.
% In a two-sided printing style, it also makes the next page a right-hand (odd-numbered) page, producing a blank page if necessary.
 


% D E C L A R A T I O N   P A G E
% -------------------------------
  % The following is the sample Delaration Page as provided by the GSO
  % December 13th, 2006.  It is designed for an electronic thesis.
  \noindent
I hereby declare that I am the sole author of this thesis. This is a true copy of the thesis, including any required final revisions, as accepted by my examiners.

  \bigskip
  
  \noindent
I understand that my thesis may be made electronically available to the public.

\cleardoublepage
%\newpage

% A B S T R A C T
% ---------------

\begin{center}\textbf{Abstract}\end{center}

Spurred by declining costs of robotics, automation is becoming a prevalent area of interest for many industries. In some cases, it even makes economic sense to use a team of robots to achieve a goal faster. In this thesis we study sweep coverage path planning, in which a robot or a team of robots must cover all points in a workspace with its footprint. In many coverage applications, including cleaning and monitoring, it is beneficial to use coverage paths with minimal robot turns.

In the first part of the thesis, we address this for a single robot by providing an efficient method to compute the minimum altitude of a non-convex polygonal region, which captures the number of parallel line segments, and thus turns, needed to cover the region. Then, given a non-convex polygon, we provide a method to cut the polygon into two pieces that minimizes the sum of their altitudes. Given an initial convex decomposition of a workspace, we apply this method to iteratively re-optimize and delete cuts of the decomposition. Finally, we compute a coverage path of the workspace by placing parallel line segments in each region, and then computing a tour of the segments of minimum cost. We present simulation results on several workspaces with obstacles, which demonstrate improvements in both the number of turns in the final coverage path and runtime.

In the second part of the thesis, we extend the results of single robot coverage to a multi robot case. We provide a metric $\chi$ that approximates the cost of a coverage path, which accounts for the cost of turns. Given a polygon, we provide a method for cutting a polygon into two that would minimize the maximum cost $\chi$ between the two polygons. Provided with an initial $n$-cell decomposition, we apply this method in the iterative manner to re-optimize cuts in order to minimize the maximum cost $\chi$ over all cells in the decomposition. For each cell in the re-optimized $n$-cell decomposition, a single robot coverage path is computed using the minimum altitude decomposition. We present the simulation results that demonstrate improvements in the maximum cost as well as the range of costs over all robots in the team.

%These examples motivate the use of robots to achieve the end goal with required performance specifications. Moreover, with the proliferation of robotics, robotic systems in general are becoming more affordable. For some applications, it's becoming economically feasible to use teams of robots to achieve a common goal. There is one problem that is common to all of the listed applications. That is the problem of the coverage path planning. That are various version of this problem with a lot of work in the literature. A solution to the problem of coverage path planning is a path for some robot that results in all point of the environment being under the robots footprint at some point along the path. This problem is known to be NP-complete and therefore, no optimal solution is known. In fact, many works in the literature propose approximate solutions that theoretically achieve the complete coverage but in practice, produce unfeasible paths.

%In this work, we develop a solution to the problem of coverage path planning, which produce resultant paths that are dynamically feasible. We aim to compute paths that consists of as many straight line segments as possible. This is motivated by the fact that straight line segments are one of the easiest paths to traverse of many robotic systems. 

%In our algorithm, we utilize special types of paths called Boustrophedon paths. These types of path are simple and facilitate theoretical analysis. We introduce a metric, which aids in the optimal orientation of Boustrophedon paths. We design a greedy algorithm that partitions a polygon with the objective of minimizing the overall altitude of a polygon. We also provide proofs of correctness and computational complexity analysis. These results are supported by simulations. 

%Finally, we extend these results to the multi-agent coverage. In the multi-agent coverage path planning problem, a team of robots with their own respective dynamics are tasked to achieve complete coverage of a polygon. We develop a partitioning scheme that divide up the workspace into regions and assign regions to each robot based on its abilities. This is to ensure that all robot perform comparable amount of work.

\cleardoublepage
%\newpage

% A C K N O W L E D G E M E N T S
% -------------------------------

\begin{center}\textbf{Acknowledgments}\end{center}

I would like to thank my advisor Stephen Smith for his guidance and endless support during my Master's career. I would also like to thank Adam Gomes and Ryan McDonald for their help and numerous insightful discussions.
\cleardoublepage
%\newpage

% D E D I C A T I O N
% -------------------
%
%\begin{center}\textbf{Dedication}\end{center}
%
%This is dedicated to the one I love.
%\cleardoublepage
%\newpage

% T A B L E   O F   C O N T E N T S
% ---------------------------------
\renewcommand\contentsname{Table of Contents}
\tableofcontents
\cleardoublepage
\phantomsection
%\newpage

% L I S T   O F   T A B L E S
% ---------------------------
\addcontentsline{toc}{chapter}{List of Tables}
\listoftables
\cleardoublepage
\phantomsection		% allows hyperref to link to the correct page
%\newpage

% L I S T   O F   F I G U R E S
% -----------------------------
\addcontentsline{toc}{chapter}{List of Figures}
\listoffigures
\cleardoublepage
\phantomsection		% allows hyperref to link to the correct page
%\newpage

% L I S T   O F   S Y M B O L S
% -----------------------------
% To include a Nomenclature section
% \addcontentsline{toc}{chapter}{\textbf{Nomenclature}}
% \renewcommand{\nomname}{Nomenclature}
% \printglossary
% \cleardoublepage
% \phantomsection % allows hyperref to link to the correct page
% \newpage

% Change page numbering back to Arabic numerals
\pagenumbering{arabic}

