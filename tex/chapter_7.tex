%!TEX root = ../main.tex

\documentclass[../main.tex]{subfiles}
\begin{document}

\appendix 


\chapter{Uncovered Area Term Derivation}
\label{appendix:area_term_derivation}

Suppose a workspace is to be covered and is populated with some straight line segments as shown in Figure~\ref{fig:sample_area}. Note the white uncovered areas near the borders of the polygon. An expression for the area of these regions is derived here. First, observe the uncovered area generated by just one line segment as shown in Figure~\ref{fig:area_error}(a), which depicts the top half of a straight segment.

Note from Figure~\ref{fig:area_error}(b), a region of interest is a trapezoid with two parallel sides $a$ and $b$, an orthogonal segment of length $2r$, and a segment of the boundary of the polygon. The area of such a trapezoid is
\begin{equation}
	A=(a+b)r.
\end{equation}
In this case, $r$ is a constant determined by the width of the coverage footprint. However, $a$ and $b$ are determined by $r$ and $\theta$, which is the angle between that particular section of the boundary and the straight line segment as shown in Figure~\ref{fig:area_error}(b). The expressions for $a$ and $b$ are as follows:
\begin{equation}
	\begin{aligned}
		&a=r\tan{\frac{\theta}{2}},\\
		&b=r\tan\left(90^o-\frac{\theta}{2}\right).
	\end{aligned}
\end{equation}

Hence, the trapezoid area can be expressed in terms of $r$ and $\theta$ as follows:
\begin{equation}
	A=2r^2\csc{\theta}.
\end{equation}
The area of a half circle is
\begin{equation}
	\frac{\pi r^2}{2}.
\end{equation}

Therefore, the amount of area that is left uncovered by the top half of the straight line segment is
\begin{equation}
	A_{\text{uncovered}}=2r^2\csc\theta-\frac{\pi r^2}{2}=\frac{r^2}{2}(4\csc\theta-\pi).
\end{equation}

\begin{figure}
	\centering
	\subfile{img/chapter_7/sample_area}
	\caption{Sample workspace with parallel line segments.}
	\label{fig:sample_area}
\end{figure}
\begin{figure}
	\centering
	\begin{subfigure}{0.5\linewidth}
		\centering
		\subfile{img/chapter_7/area_error}
		\caption{\label{fig:area_error_i}}
	\end{subfigure}%
	\begin{subfigure}{0.5\linewidth}
		\centering
		\subfile{img/chapter_7/area_error_i}
		\caption{\label{fig:area_error_ii}}
	\end{subfigure}
	\caption{Geometry of the uncovered area.}
	\label{fig:area_error}
\end{figure}

Note that the expression for $A_{\text{uncovered}}$ is exact and $A_{\text{uncovered}}$ is a function of $\theta$, where $\theta$ varies from one edge of a polygon to the next and the orientation of the straight line segments themselves. As such, the computation of this term requires the knowledge of $\theta$. Next, we show an approximation to this value that gets rid of this requirement.

Suppose we compute $\theta_{\max}$ where
\begin{equation}
	\theta_{\max}=\arg\max_{\theta\in\Theta}\left(\frac{r^2}{2}(4\csc\theta-\pi)\right).
\end{equation}%
Here, $\Theta$ is a set of all possible $\theta$ that a set of lines can have with the respect to any edge of a polygon. Fortunately, by the result from Huang~\cite{Huang2001optimal}, the optimal orientation of a set of lines is orthogonal to one of the edges of a polygon. Hence, if there are $n$ edges in a polygon then $|\Theta|=n^2$.

Hence, the uncovered area can then be approximated as follows:
\begin{equation}
	A_{\text{uncovered}}^{\max}=2|S|\frac{r^2}{2}(4\csc\theta_{\max}-\pi).
\end{equation}%
Here, $|S|$ is the number of straight line segments in a workspace scaled by a factor of two to account for both ends of a line. Although $|S|$ is unknown prior to computing the coverage path, it can be approximated by the minimum altitude. This approximation is exact for convex polygons and is an over approximation for non-convex polygons. Therefore, the term is modified as follows:
\begin{equation}
	A_{\text{uncovered}}^{\max}=\ceil{\frac{\alpha_{\min}}{r}}r^2(4\csc\theta_{\max}-\pi).
\end{equation}

\end{document}