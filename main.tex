%\documentclass[a4paper,11pt]{article}
\documentclass[letterpaper,12pt,titlepage,oneside,final]{book}

\makeatletter
\def\input@path{{../../}{../}{./}}
\makeatother

\usepackage{mystyle}

%======================================================================
%   L O G I C A L    D O C U M E N T -- the content of your thesis
%======================================================================
\begin{document}

% For a large document, it is a good idea to divide your thesis
% into several files, each one containing one chapter.
% To illustrate this idea, the "front pages" (i.e., title page,
% declaration, borrowers' page, abstract, acknowledgements,
% dedication, table of contents, list of tables, list of figures,
% nomenclature) are contained within the file "uw-ethesis-frontpgs.tex" which is
% included into the document by the following statement.
%----------------------------------------------------------------------
% FRONT MATERIAL
%----------------------------------------------------------------------
% T I T L E   P A G E
% -------------------
% Last updated May 24, 2011, by Stephen Carr, IST-Client Services
% The title page is counted as page `i' but we need to suppress the
% page number.  We also don't want any headers or footers.
\pagestyle{empty}
\pagenumbering{roman}

% The contents of the title page are specified in the "titlepage"
% environment.
\begin{titlepage}
        \begin{center}
        \vspace*{1.0cm}

        \Huge
        {\bf Minimizing Turns in Single and Multi Robot Coverage Path Planning}

        \vspace*{1.0cm}

        \normalsize
        by \\

        \vspace*{1.0cm}

        \Large
        Stanislav Bochkarev \\

        \vspace*{3.0cm}

        \normalsize
        A thesis \\
        presented to the University of Waterloo \\ 
        in fulfillment of the \\
        thesis requirement for the degree of \\
        Master of Applied Science \\
        in \\
        Electrical and Computer Engineering \\

        \vspace*{2.0cm}

        Waterloo, Ontario, Canada, 2016 \\

        \vspace*{1.0cm}

        \copyright\ Stanislav Bochkarev 2016 \\
        \end{center}
\end{titlepage}

% The rest of the front pages should contain no headers and be numbered using Roman numerals starting with `ii'
\pagestyle{plain}
\setcounter{page}{2}

\cleardoublepage % Ends the current page and causes all figures and tables that have so far appeared in the input to be printed.
% In a two-sided printing style, it also makes the next page a right-hand (odd-numbered) page, producing a blank page if necessary.
 


% D E C L A R A T I O N   P A G E
% -------------------------------
  % The following is the sample Delaration Page as provided by the GSO
  % December 13th, 2006.  It is designed for an electronic thesis.
  \noindent
I hereby declare that I am the sole author of this thesis. This is a true copy of the thesis, including any required final revisions, as accepted by my examiners.

  \bigskip
  
  \noindent
I understand that my thesis may be made electronically available to the public.

\cleardoublepage
%\newpage

% A B S T R A C T
% ---------------

\begin{center}\textbf{Abstract}\end{center}
There is a growing number of applications in our lives that rely on robots, which leads to cheaper, smarter, more reliable robotic systems. However, there are still numerous applications where humans cannot be utilized; either because the environment is too hostile or humans are not quick, precise, or reliable enough to make them economically viable agents. Some examples of applications with hostile environments include mine sweeping, sweeping ocean floor for debris, search and rescue in the wilderness, natural disaster rescue efforts, and nuclear meltdown containment efforts. Here, the situation is too risky to let humans operate in those conditions. Some examples of applications with tight economic are factory paint finish, industrial floor sweeping, CNC drilling, and crop monitoring among others. And here, the speed and precision of humans is often not good enough to create the end product in a quick enough effort to meet the cost requirement.

These examples motivate the use of robots to achieve the end goal with required performance specifications. Moreover, with the proliferation of robotics, robotic systems in general are becoming more affordable. For some applications, it's becoming economically feasible to use teams of robots to achieve a common goal. There is one problem that is common to all of the listed applications. That is the problem of the coverage path planning. That are various version of this problem with a lot of work in the literature. A solution to the problem of coverage path planning is a path for some robot that results in all point of the environment being under the robots footprint at some point along the path. This problem is known to be NP-complete and therefore, no optimal solution is known. In fact, many works in the literature propose approximate solutions that theoretically achieve the complete coverage but in practice, produce unfeasible paths.

In this work, we develop a solution to the problem of coverage path planning, which produce resultant paths that are dynamically feasible. We aim to compute paths that consists of as many straight line segments as possible. This is motivated by the fact that straight line segments are one of the easiest paths to traverse of many robotic systems. 

In our algorithm, we utilize special types of paths called Boustrophedon paths. These types of path are simple and facilitate theoretical analysis. We introduce a metric, which aids in the optimal orientation of Boustrophedon paths. We design a greedy algorithm that partitions a polygon with the objective of minimizing the overall altitude of a polygon. We also provide proofs of correctness and computational complexity analysis. These results are supported by simulations. 

Finally, we extend these results to the multi-agent coverage. In the multi-agent coverage path planning problem, a team of robots with their own respective dynamics are tasked to achieve complete coverage of a polygon. We develop a partitioning scheme that divide up the workspace into regions and assign regions to each robot based on its abilities. This is to ensure that all robot perform comparable amount of work.

\cleardoublepage
%\newpage

% A C K N O W L E D G E M E N T S
% -------------------------------

\begin{center}\textbf{Acknowledgments}\end{center}

I would like to thank my advisor Stephen Smith for his guidance and endless support during my Master's career. 
\cleardoublepage
%\newpage

% D E D I C A T I O N
% -------------------
%
%\begin{center}\textbf{Dedication}\end{center}
%
%This is dedicated to the one I love.
%\cleardoublepage
%\newpage

% T A B L E   O F   C O N T E N T S
% ---------------------------------
\renewcommand\contentsname{Table of Contents}
\tableofcontents
\cleardoublepage
\phantomsection
%\newpage

% L I S T   O F   T A B L E S
% ---------------------------
\addcontentsline{toc}{chapter}{List of Tables}
\listoftables
\cleardoublepage
\phantomsection		% allows hyperref to link to the correct page
%\newpage

% L I S T   O F   F I G U R E S
% -----------------------------
\addcontentsline{toc}{chapter}{List of Figures}
\listoffigures
\cleardoublepage
\phantomsection		% allows hyperref to link to the correct page
%\newpage

% L I S T   O F   S Y M B O L S
% -----------------------------
% To include a Nomenclature section
% \addcontentsline{toc}{chapter}{\textbf{Nomenclature}}
% \renewcommand{\nomname}{Nomenclature}
% \printglossary
% \cleardoublepage
% \phantomsection % allows hyperref to link to the correct page
% \newpage

% Change page numbering back to Arabic numerals
\pagenumbering{arabic}

 

%----------------------------------------------------------------------
% MAIN BODY
%----------------------------------------------------------------------
\subfile{tex/chapter_1}
\subfile{tex/chapter_2}

\begin{figure}
	\centering
	\subfile{img/chapter_2/workspace_and_system}
	\caption{The workspace polygon and a point model of the robot.}
	\label{fig:workspace_and_system}
\end{figure}

\begin{figure}
	\centering
	\subfile{img/chapter_2/configuration_space}
	\caption{An example of a configuration space for a point robot.}
	\label{fig:configuration_space}
\end{figure}

\begin{figure}
	\centering
	\subfile{img/chapter_2/coverable_space}
	\caption{An example of a coverable space.}
	\label{fig:coverable_space}
\end{figure}


\section{Problem Statement}
\label{sec:problem_statement}

\subfile{tex/chapter_3}
\subfile{tex/chapter_4}

\section{Single Robot Coverage}
\label{sec:single_robot_coverage}

%\subfile{tex/chapter_5}

\begin{figure}
	\centering
	\subfile{img/chapter_3/pinch_vertex}
	\caption{An example of a pinch vertex at $v_p$.}
	\label{fig:pinch_vertex}
\end{figure}
\begin{figure}
	\centering
	\subfile{img/chapter_3/transition_point}
	\caption{An example of a transition point.}
	\label{fig:transition_point}
\end{figure}
\begin{figure}
	\centering
	\subfile{img/chapter_3/altitude}
	\caption{Process of measuring altitude with $\theta$ equal to 0.}
	\label{fig:altitude}
\end{figure}
\begin{figure}
	\centering
	\begin{subfigure}{0.5\linewidth}
		\centering
		\subfile{img/chapter_3/reorient_solution_1}
	\end{subfigure}%
	\begin{subfigure}{0.5\linewidth}
		\centering
		\subfile{img/chapter_3/reorient_solution_2}
	\end{subfigure}
	\caption{Polygon regions where lines can be reoriented.}
	\label{fig:reorder_regions}
\end{figure}
\begin{figure}
	\centering
	\begin{subfigure}{0.45\linewidth}
		\centering
		\subfile{img/chapter_3/altitude_func_pl_1}
		\caption{\label{fig:altitude_case_i}}
	\end{subfigure}%
	\quad
	\begin{subfigure}{0.45\linewidth}
		\centering
		\subfile{img/chapter_3/altitude_func_pl_2}
		\caption{\label{fig:altitude_case_ii}}
	\end{subfigure}
	\begin{subfigure}{0.45\linewidth}
		\centering
		\subfile{img/chapter_3/altitude_func_pl_2}
		\caption{\label{fig:altitude_case_iii}}
	\end{subfigure}
	\caption{Three cases of altitude behavior for $P_{\ell}$.}
	\label{fig:altitude_cases_pl}
\end{figure}

\begin{figure}
	\centering
	\begin{subfigure}{0.45\linewidth}
		\centering
		\subfile{img/chapter_3/altitude_func_pr_1}
		\caption{\label{fig:altitude_case_pr_i}}
	\end{subfigure}%
	\quad
	\begin{subfigure}{0.45\linewidth}
		\centering
		\subfile{img/chapter_3/altitude_func_pr_2}
		\caption{\label{fig:altitude_case_pr_ii}}
	\end{subfigure}
	\begin{subfigure}{0.45\linewidth}
		\centering
		\subfile{img/chapter_3/altitude_func_pr_3}
		\caption{\label{fig:altitude_case_pr_iii}}
	\end{subfigure}
	\caption{Three cases of altitude behavior for $P_r$.}
	\label{fig:altitude_cases_pr}
\end{figure}


\section{Multi-agent Coverage}
\label{sec:multi-agent_coverage}

%\subfile{tex/chapter_6}
\begin{figure}
	\centering
	\subfile{img/chapter_4/spiral_triangle_coverage}%
	\subfile{img/chapter_4/spiral_triangle_coverage_1}%
	\subfile{img/chapter_4/spiral_triangle_coverage_2}
	\caption{A triangle to be completely covered.}
	\label{fig:triangle_1}
\end{figure}

%----------------------------------------------------------------------
% END MATERIAL
%----------------------------------------------------------------------

% B I B L I O G R A P H Y
% -----------------------

% The following statement selects the style to use for references.  It controls the sort order of the entries in the bibliography and also the formatting for the in-text labels.
\bibliographystyle{plain}
% This specifies the location of the file containing the bibliographic information.  
% It assumes you're using BibTeX (if not, why not?).
\cleardoublepage % This is needed if the book class is used, to place the anchor in the correct page,
                 % because the bibliography will start on its own page.
                 % Use \clearpage instead if the document class uses the "oneside" argument
\phantomsection  % With hyperref package, enables hyperlinking from the table of contents to bibliography             
% The following statement causes the title "References" to be used for the bibliography section:
\renewcommand*{\bibname}{References}

% Add the References to the Table of Contents
\addcontentsline{toc}{chapter}{\textbf{References}}

\bibliography{main}
% Tip 5: You can create multiple .bib files to organize your references. 
% Just list them all in the \bibliogaphy command, separated by commas (no spaces).

% The following statement causes the specified references to be added to the bibliography% even if they were not 
% cited in the text. The asterisk is a wildcard that causes all entries in the bibliographic database to be included (optional).
\nocite{*}

\end{document}